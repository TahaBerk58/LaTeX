\document{article}
\usepackage{graphicx, xcolor}

\title{} %Başlık
\author{} %Yazar
\thanks{text} %Sorumlu Yazar
\date{text} %Tarih

\begin{document}

\maketitle

\section{Giriş} %Giriş başlığı oluşturur.

Buraya text yazılır.

\textit{text} %İtalik yazım
\textbf{text} %Bold yazım
\textcolor{red}{text}
\underline{text} %Altı çizgili
\emph{text} %Metin vurgusu
\newline{text} %Yeni satıra geç

\section{Düzenleme}

Text...

\begin{itemize} %Noktalı madde
    \item A
    \item B
    \item C
\end{itemize}

\begin{enumerate}
    \item A
    \item B
    \item C
\end{enumerate}

\subsection{Şekil Ekleme}

\begin{Figure}[h!] %H resmi kodu yazdığın here koyar.
    \centering
    \includegraphics[width=0.6 \textwidth]{"Dosya Yolu"}
    \caption{YTÜ}
    \label{fig:ytu_logo}
\end{figure}

Text... \ref{fig:ytu_logo}

\subsection{Math}

Cümle içinde denklem kullanımı $E=mc^2$ \ref{eq:kütle}

\begin{equation} \label{eq:kütle}
    E=mc^2
\end{equation}

\subsection{Tablo Eklene} \label{sec:tablo}

\begin{center}
    \begin{tabular}{c, c, c}
        cell 1 & cell 2 & cell 3 \\
        cell 4 & cell 5 & cell 6 \\
        cell 7 & cell 8 & cell 9 \\
    \end{tabular}
\end{center}


\end{document}